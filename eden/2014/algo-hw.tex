\documentclass{article}
\usepackage{amssymb}
\usepackage{amsmath}
\usepackage{amsfonts}
\usepackage{latexsym}
\usepackage{times}
%\usepackage{psfrag,psfig,epsfig,epsf}
\usepackage{graphics}
\usepackage{multirow}
\usepackage{fullpage}
\usepackage{verbatim}
\usepackage{fancyheadings}
\usepackage[T1]{fontenc}
\usepackage{arev}
\usepackage{subfigure}
\usepackage{url}
\usepackage[noline,noend,ruled,linesnumbered]{algorithm2e}
\usepackage{algpseudocode}
\linespread{1.02} 

\pagestyle{empty}

\addtolength{\topmargin}{-20pt}
\addtolength{\oddsidemargin}{-5pt}
\addtolength{\textwidth}{20pt}
\addtolength{\textheight}{50pt}

\newenvironment{myitem}{\begin{list}{$\bullet$}
{\setlength{\itemsep}{-0pt}
\setlength{\topsep}{0pt}
\setlength{\labelwidth}{0pt}
%\setlength{\labelsep}{0pt}
\setlength{\leftmargin}{10pt}
\setlength{\parsep}{-0pt}
\setlength{\itemsep}{0pt}
\setlength{\partopsep}{0pt}}}%
{\end{list}}

\begin{document}

\sloppy

\noindent \underline{CS 344: DESIGN AND ANALYSIS OF COMPUTER
  ALGORITHMS \hspace{1.6in} SPRING 2014}

\vspace{0.1in}

\begin{center}
{\bf {\large Homework 1}}\\
Asymptotics and Number Theoretic Algorithms\\
\end{center}

\vspace{0.1in}

\noindent Deadline: February 18, 11:59pm.\\ 
Available points: 120. Perfect score: 100.\\

\begin{center}
{\bf Homework Instructions:}
\end{center}

\noindent {\bf Teams:} Homeworks should be completed by teams of
students - three at most. No additional credit will be given for
students that complete a homework individually. Please inform
Athanasios Krontiris about the members of your team as soon as
possible (email: ak979 AT cs.rutgers.edu).\\

\noindent {\bf Submission Rules:} Submit your solutions electronically
as a PDF document through \url{sakai.rutgers.edu}. Do not submit Word
documents, raw text, or hardcopies etc. Make sure to generate and
submit a PDF instead. Each team of students should submit only a
single copy of their solutions and indicate all team members on their
submission.  Failure to follow these rules will result in lower grade
in the assignment.\\

\noindent {\bf Late Submissions:} No late submission is allowed. If
you don't submit a homework on time, you get 0 points for that
homework.\\

\noindent {\bf Extra Credit for \LaTeX:} You will receive 10\% extra
credit points if you submit your answers as a typeset PDF (using
\LaTeX, in which case you should also submit electronically your
source code). Resources on how to use \LaTeX\ are available on the
course's website. There will be a 5\% bonus for electronically
prepared answers (e.g., on MS Word, etc.) that are not typeset.\\

\noindent {\bf 25\% Rule:} For any homework problem (same will hold
for exam questions), you can either attempt to answer the question, in
which case you will receive between 0 and 100\% credit for that
question, or you can write "I don't know", in which case you receive
25\% credit for that question. Leaving the question blank is the same
as writing "I don't know." You can and will get less than 25\% credit
for a question that you answer erroneously.\\

\noindent {\bf Handwritten Reports:} If you want to submit a
handwritten report, scan it and submit a PDF via Sakai. We will not
accept hardcopies. If you choose to submit handwritten answers and we
are not able to read them, you will not be awarded any points for the
part of the solution that is unreadable.\\

\noindent {\bf Precision:} Try to be precise. Have in mind that you
are trying to convince a very skeptical reader (and computer
scientists are the worst kind...) that your answers are correct.\\

\noindent {\bf Collusion, Plagiarism, etc.:} Each team of students
must prepare its solutions independently from other teams, i.e.,
without using common notes or worksheets with other students or trying
to solve problems in collaboration with other teams.  You must
indicate any external sources you have used in the preparation of your
solution. Do not plagiarize online sources and in general make sure
you do not violate any of the academic standards of the course, the
department or of the university (the standards are available through
the course's website). Failure to follow these rules may result in
failure in the course.\\


\newpage

\vspace{0.1in}

{\bf }

\begin{center}
{\bf Part A (20 points)}
\end{center}

\noindent {\bf Problem 1:} In each of the following situations
indicate whether $f=O(g)$ or $f=\Omega(g)$ or $f=\Theta(g)$:

\begin{enumerate}
\item $f(n) = \sqrt{2^{7n}},\ g(n) = \lg({7^{2n}})$
\item $f(n) = 2^{nln(n)},\ g(n) = n!$
\item $f(n) = lg(lg^*n),\  g(n) = lg^*(lgn)$
\item $f(n) = \frac{lgn^2}{n},\ g(n) = lg^*n$
\item $f(n) = 2^n,\ g(n) = n^{lgn}$
\item $f(n) = 2^{\sqrt{lgn}},\ g(n) = n(lgn)^3$
\item $f(n) = e^{cos(n)},\ g(n) = lgn$
\item $f(n) = lgn^2,\ g(n) = (lgn)^2$
\item $f(n) = \sqrt{4n^2 - 12n + 9},\ g(n) = n^{\frac{3}{2}}$
\item $f(n) = \sum_{k=1}^{n} k, g(n) = (n+2)^2$
\end{enumerate}

\noindent \emph{Note:} The $lg^*n$ function is the number of times the
logarithm function must be iteratively applied before the result is
less than or equal to 1, i.e., $lg^*n = 0$ if $n \leq 1$ and $lg^*n =
1 + lg^(lgn)$ if $n > 1$. For instance, $lg^*4=2$, $lg^*16=3$,
$lg^*65536=4$, etc.\\

\noindent \emph{Note:} To check the relative asymptotic behavior
between two functions $f(n)$ and $g(n)$, we may use transformation by
a monotonically increasing function. For instance, we usually use
logarithmic functions for simplifying the expressions. This is because
logarithmic functions convert multiplication to addition and power to
multiplication. Nevertheless, you can apply any monotonically
increasing function in the range from 0 to infinity to simplify the
expressions. If after the application of the transformation, one
expression dominates the other as $n$ goes to infinity, then this is
also true for the original expressions. Examples include a square
function $x^2$ or even an exponential function $e^x$. For instance, to
compare $f(n)$ and $g(n)$, it is sufficient to compare $f^2(n)$ and
$g^2(n)$.

\begin{algorithm}[h]
\caption{${\tt Number\_Theoretic\_Algorithm}$ $($ integer $n$ ) }
\label{algo}
$N \leftarrow Random\_Sample( 0, 2^n-1)$\;
\If{ $N$ is even }
{
  $N \leftarrow N + 1$\;
}
$m \leftarrow N \mod\ n$\;
\For{ $j \leftarrow 0$ to $m$ }
{
  \If{ ${\tt Greatest\_Common\_Divisor}$( $j, N ) \neq 1$ }
  {
    return {\tt FALSE};
  }
  Compute $x,z$ so that $N - 1 = 2^z \cdot x$ and $x$ is odd\;
  $y_0 \leftarrow (N-1-j)^x \mod N$\;
  \For{ $i \leftarrow 1$ to $m$}
  {
    $y_i \leftarrow y_{i-1}^2 \mod N$\;
    $y_i \leftarrow y_i + y_{i-1} \mod N$\;
  }
  \If{ {\tt Low\_Error\_Primality\_Test}($y_m$) == {\tt FALSE} }
  {
    return {\tt FALSE}\;
  }
}
return {\tt TRUE}\;
\end{algorithm}

\newpage

\noindent {\bf Problem 2:} Compute the asymptotic running time of the
above algorithm as a function of its input parameter, given:

\begin{myitem}
\item The running times of integer arithmetic operations (e.g.,
  multiplication of two large $n$-bit numbers is $O(n^2)$).
\item Assume that sampling a number $N$ is an operation linear to the
  number of bits needed to represent this number.
\end{myitem}

\noindent Do not just present the final result. For each line of
pseudo-code indicate the best running time for the corresponding
operation given current knowledge from lectures and recitations and
then show how the overall running time emerges.\\

\begin{center}
{\bf Part B (30 points)}
\end{center}

\noindent {\bf Problem 3:} 
\begin{myitem}
\item Consider that we have a tree data structure $T_m^N$, where every
  node can have at most $m$ children and the tree has at most $N$
  nodes total. Compute a lower bound for the height of the tree.\\

\item Consider two such trees $T_m^N$ and $T_{m'}^N$ that are
  ``perfect'', i.e., every node has exactly $m$ and $m'$ children
  correspondingly. Now, consider the functions $h_m(N)$ and
  $h_{m'}(N)$ that express the heights of these perfect trees for
  different values of $N$. What is the asymptotic behavior of $h_m$
  relative to $h_{m'}$ and under what conditions?\\

\item Consider the following rule for modular exponentiation modulo a
  number $N$, where $x$ is in the order of $2^m$, $y$ is in the order
  of $2^n$ and $N$ is in the order of $2^o$. What is the running time
  of computing the result according to this rule?
$$x^y \mod N \equiv \left\{
\begin{array}{c l}     
    (x ^{\lfloor \frac{y}{2} \rfloor} )^2 \mod N,  & \textrm{ if y is even }\\
    x \cdot (x^{\lfloor \frac{y}{2} \rfloor} )^2 \mod N, & \textrm{ if y is odd}
\end{array}\right.$$
\end{myitem}

\noindent {\bf Problem 4:}
\begin{myitem}
\item Compute the following: $2^{902}$ mod 7.\\
\item Find the modulo multiplicative inverse of 11 mod 120, 13 mod 45,
  35 mod 77, 9 mod 11, 11 mod 1111.\\
\item Assume that for a number $x$ the following property is true:
  $\forall\ y \in [1,x-1]:\ gcd(x,y) = 1$. Compute the running time of
  an efficient algorithm for finding all the inverses modulo $x^m$
  from the set $\{0, 1, \ldots, x^m-1\}$ that exist.\\
\end{myitem}

\noindent {\bf Problem 5:} 
\begin{myitem}
\item Assume two positive integers x < y. Then the pairs
  $(5x+3y,3x+2y)$ and $(x,y)$ have the same greater common
  divisor. True or False, explain.\\

\item Consider the following sequence of numbers: $s_n = 1 +
  \Pi_{i=0}^{n-1} s_i$, where $s_0 = 2$. Prove that any two numbers in
  this sequence are relatively prime.\\
\end{myitem}

\begin{center}
{\bf Part C (40 points)}
\end{center}

\noindent {\bf Problem 6:} Our goal is to assign $(2^{31}-1)$ integers
into 256 slots $\{0,1, \dots, 255\}$ and achieving the properties of
universal hashing. Notice that 256 is not a prime number. Assume the
following approach towards this objective.\\

\noindent Consider a 32-bit integer $y$ and the following set $\mathcal{M}$ of
hash functions, so that:
\begin{myitem}
\item each $m \in \mathcal{M}$ corresponds to a unique $8 \times 32$
  matrix $M$ having elements only 0 or 1.
\item and $m(y) = M \cdot y$ mod $2$, i.e., multiply the matrix with
  the 32-bit vector corresponding to number $y$ and then apply the
  modulo operation on each bit of the resulting vector.\\
\end{myitem}

\noindent Such functions map a 32-bit vector into an 8-bit vector,
which can then be interpreted as an 8-bit number that indicates the
original number's slot in the range $0 \sim 255$. Notice that there
are $2^{8\times32}=2^{256}$ different $\{0,1\}$ matrices in the set
$\mathcal{H}$.\\

\noindent Provide the following:
\begin{myitem}
\item A proof that the hash function family $\mathcal{M}$ is
  universal. [Hint: You have to show that the hash function family has
    the ``consistent'' and ``random'' property. The first is rather
    obvious to show. The second requires that before fixing the matrix
    $M$ the probability of two distinct vectors $x$ and $y$ to be
    mapped to the same index $m(x) = m(y)$ is equal to the probability
    of them assigned to the same index, when indices $m(x)$ and $m(y)$
    are sampled uniformly at random. So you need to consider how two
    different vectors $x$ and $y$ can be mapped to the same index
    $m(x)$ and $m(y)$ when $m(x) \equiv Mx \mod 2$ and $m(y) \equiv
    My \mod 2$.]

\item A comparison to the universal hash function family described in
  DPV chapter 1.5.2. How many random bits are needed here?\\
\end{myitem}

\noindent {\bf Problem 7:} Answer the following sequence of problems:
\begin{itemize}

\item Assume that number $n$ is prime, then all numbers $1 \leq x < n$
  are invertible modulo $n$. Which of these numbers are their own
  inverse modulo $n$?  [Hint: $x$ is its own inverse modulo $n$ if
    $x^{2} \equiv 1 \mod n$ which means $x^{2}-1 \equiv 0 \mod
    n$. Note that when a number $k$ can be written as the product $k =
    k_1 \cdot k_2$, then the factorization of $k$ corresponds to the
    product of the factorizations of $k_1$ and $k_2$. The
    factorization of a number $k$ corresponds to the product of primes
    that result in $k$. All these primes must be smaller then $k$.]

\item Show that $(n-1)! \equiv -1 \mod n$ for prime $n$. [Hint:
  Consider all the pairs of numbers that are smaller than $n$, which
  are multiplicative inverses modulo $n$, multiply these numbers and
  compute the result of this product modulo $n$. You have to show in
  this process that the multiplicative inverse of a number $x$ exists
  and is unique. Try then to generate the $(n-1)!$ on the left side
  and see what you get on the right side.]

\item Show that if $n$ is not prime, then $(n-1)! \neq -1 \mod
  n$. [Hint: What does it mean that $n$ is not prime in terms of the
    numbers $1 \leq x < n$? If you try to compute $(n-1)!$ the same
    way as in the above question given that $n$ is not prime, what
    will happen?]

\item The above process can be used as a primality test instead of
  Fermat's Little theorem as it is an if-and-only-if condition for
  primality. Why can't we immediately base a primality test on this
  rule? [Tip: Even if you are not able to answer the previous two
    questions, you should still be able to argue about this
    question.]\\

\end{itemize}

\begin{center}
{\bf Part D (30 points)}
\end{center}


\noindent {\bf Problem 8:}

\noindent A. Make a table with three columns. The first column is all numbers
from 0 to 36. The second is the residues of these numbers modulo 5;
the third column is the residues modulo 7.\\

\noindent B. Consider two different prime numbers $x$ and $y$. Show that the
following is true: for every pair of numbers $m$ and $n$ so that: $0
\leq m < x$ and $0 \leq n < y$, there is a unique integer $q$, where
$0 \leq q < xy$, so that:

$$q \equiv m \pmod{x}$$
$$q \equiv n \pmod{y}$$

[Hint: Think how many $q$'s in the range $[0,xy]$ can have the same
  result modulo $x$ and modulo $y$ and count how many $q$'s there
  are.]\\
  
\noindent C. The previous problem asks to go from $q$ to $(m,n)$. It is also
possible to go the other way. In particular, show the following:

$$q = ( m \cdot y \cdot (y^{-1} \bmod{x}) + n \cdot x \cdot
(x^{-1} \bmod{y}) ) \bmod{xy}$$

[Hint: Ensure that if the above is true then the expressions in section B are
  also true. Consider the values of the following terms: $c_x = y
  \cdot (y^{-1} \bmod{x})$ and $c_y = x \cdot (x^{-1} \bmod{y})$ and
their values $\mod{x}$ and $\mod{y}$.]\\

\noindent D. What happens in the case of three primes $x$, $y$ and $z$? Do the
above properties still hold? If they do, how do they look like in this
case?\\
  
\noindent {\bf Problem 9:} There is an office, in which every member
uses RSA to conduct secure communication with others.  For example,
when Alice wants to send Bob a message, she will first use Bob's
public key to encrypt the message, then sends it to Bob, Bob then uses
his private key to decrypt the message.  In order to make things easy,
the office maintains a directory listing every member's public key,
everybody in the office has access to it. A public key looks like $(N,
e)$ (as defined in DPV-1.4.2).\\

\noindent One day, Alice sent a message (smaller than any $N$) to Bob,
Charlie, and David via the RSA based secure communication, but the
encrypted messages were intercepted by the webmaster Mallory. Mallory
used some network tricks and found out who the receivers were, then he
checked their public keys in the directory. This is what Mallory got:

\begin{center}
  \begin{tabular}{ | c | c | c | }
    \hline
    Receiver & Encrypted message & Public key \\ \hline
    Bob & 674 & (3337, 3) \\
    Charlie & 36 & (187, 3) \\
    David & 948 & (1219, 3) \\
    \hline
  \end{tabular}
\end{center}

\noindent Finally, Mallory recovered the original message. How did
Mallory do this? What is the original message?\\

\noindent \emph{Note:} Please do not directly use factorization. It is
possible to do so because the numbers are small. You can use
factorization to validate your result.\\

\noindent [Hint: Consider how the three transmitted messages are
  computed. The result of the previous problem 8 is useful here. Be
  careful with the details and under which conditions you apply
  certain expressions.]\\

\end{document}

